% A UCalgary-branded poster using tikzposter

\documentclass{tikzposter} % Options for format can be included here

% theme inspired by UCalgary-branded poster
\usepackage{durhamuni-poster}

% Make the size 60 by 42"
\geometry{paperwidth=841mm,paperheight=1188mm}
\tpsetpostersize

% Make the body font a bit larger (33pt)
\def\large{\fontsize{33}{40}\selectfont}

% Why not have links?
\usepackage[hidelinks]{hyperref}

\usepackage{amsmath}
%\usepackage[colalign]{colalign}

% Title, Author, Institute

\title{Extending Peano4: coupling solvers for systems of PDEs}
\author{Dmitry Nikolaenko}
\institute{Durham University}

\begin{document}

 % Title block with title, author, logo, etc.
\maketitle

% Layout: we use 4 columns, each .25 of textwidth
\begin{columns}

% 1st column
\column{0.3}
\block{Introduction}{
    \textbf{Peano} is a \textit{framework} for PDE solvers on hierarchical meshes with adaptive mesh refinement \cite{ref1} and provides the mesh management, data storage, distribution and mesh traversal on the Peano \textit{space-filling curve}. Peano is massively parallelised and capable of exploiting the full resources of supercomputers. Its current version 4 is shipped with several \textit{specialised extensions} following Peano architecture as well as applications and benchmarks to demonstrate its work and assess its performance.
    %These extensions rely on the unified API provided by Peano4.
    \vspace{0.5in}

    \textbf{ExaHyPE2} is one of \underline{extensions} for solving systems of first-order hyperbolic partial differential equations (PDEs) \cite{REINARZ2020107251} (e.g., in problems of seismology and astrophysics). It provides a generic engine and collection of solvers (such as finite volume method and higher order ADER discontinuous Galerkin schemes) to solve systems of PDEs.
    \vspace{0.5in}

    %\textbf{Swift2} is another \underline{extension} for solving astrophysical problem by means of Smoothed Particle Hydrodynamics (SPH) method. It integrates the SPH kernels from SWIFT software employing task parallelism with Peano's particle-in-dual-tree concept and its particles toolbox.
    %\vspace{0.5in}

    \textbf{Multigrid} is another \underline{extension} being developed in collaboration with mathematicians from Bath University. It implements elliptic solvers based on multigrid methods using a hierarchy of discretisations.
    \vspace{0.5in}

    Every Peano application, including ones based on the above extensions, is a C++ code which follows the unified Peano architecture consisting of several layers:
    \begin{itemize}
        \item Technical Architecture \texttt{<<tarch>>}
        \item Peano core \texttt{<<peano4>>}
        \item Various toolboxes, on top of which extensions are built, e.g.:
        \begin{itemize}
            \item \texttt{<<toolbox::blockstructured>>} for blockstructured meshes
            \item \texttt{<<toolbox::particles>>} for particle management
            \item \texttt{<<toolbox::loadbalancing>>} for dynamic load balancing
        \end{itemize}
    \end{itemize}
}

% 2nd column
\column{0.7}

\begin{subcolumns}
    \subcolumn{0.6}
    \block{}{
        \centering
        \includegraphics[width=0.55\colwidth]{images/architecture-static.png}
    }

    \subcolumn{0.4}
    \block{Motivation}{
    \begin{minipage}[t]{\linewidth}
        The Python API simplifies creating applications by writing python scripts within the framework of one Peano extensions.
        \begin{itemize}
            \item \textbf{Old approach} Pick one extension suitable for the given problem (reformulate the problem if necessary) and write within it
            \item \textbf{New approach} Identify parts of the given complex problem suitable for solving by different extensions, implement them each within its extension (e.g. solve hyperbolic equations in \textit{ExaHyPE} and elliptic in \textit{Multigrid}) and generate an application coupling solvers from both
        \end{itemize}
    \end{minipage}
}

\end{subcolumns}

\begin{subcolumns}
    \subcolumn{0.5}
    \block{Example problem formulation}{}
    \block{}{
        \underline{Hyperbolic problem} for Euler equation:

        \textit{<<Euler system of PDEs>>}
    }
    \block{}{
        \underline{Dirichlet problem} for Poisson equation:
        \begin{equation*}
            \begin{cases}
                \Delta u = -f(x,y) & \text{in } \Omega, \\
                u = 0 & \text{on } \partial \Omega,
            \end{cases}
        \end{equation*}
    }

    \subcolumn{0.5}
    \block{General workflow}{
        (using an example of a Peano application based on \textit{Multgrid} extension):
        \begin{enumerate}
            \item Create an \texttt{mghype} Project
            \item Construct matrices
            \item Instantiate solvers and add them to the Project
            \item Configure the Project
            \item Generate a \texttt{peano4} Project
        \end{enumerate}
    }
\end{subcolumns}

\block{Problem}{
    Implement a coupling between a hyperbolic solver (provided by \textit{ExaHyPE}) for the Euler equation and an elliptic solver (provided by \textit{Multigrid}) for the Poisson equation.
}

\begin{subcolumns}
    \subcolumn{0.4}
    \begin{subcolumns}
    \subcolumn{0.4}
    \block{Solution}{
        \begin{enumerate}
            \item Generate the 1st \textit{Peano} application from one extension, e.g. \textit{Multigrid}, following the same workflow as above
            \item Generate the 2nd \textit{Peano} application from another extension, e.g. \textit{ExaHyPE}, in a similar fashion
            \item (new feature) Merge the above 2 \texttt{<<peano4>>} Projects coupling solvers from both
        \end{enumerate}
    }

    \subcolumn{0.6}
    \block{Discussion}{
        In line with the theme proposed for \textit{RSECon24}, this work tries to follow some guiding principles of FAIR for Research Software:
        \begin{itemize}
            \item New ways of coupling solvers for different PDE systems are realised which weren't possible before, thus improving \textit{interoperability}. Various Peano extensions, such as \textit{ExaHyPE} and \textit{Multigrid}, are integrated.
            \item In turn, \textit{reusability} of the extensions as building blocks is enhanced.
        \end{itemize}
    }

\end{subcolumns}


    \subcolumn{0.6}
    \block{Discussion}{
    In line with the theme proposed for RSECon24, My RSE work tries to follow some guiding principles of FAIR for Research Software:
    \begin{itemize}
        \item My implementation enables new ways of coupling solvers for different PDE systems which weren't possible before, thus improving \textit{interoperability}. Various Peano4 extensions, such as \textit{ExaHyPE} and \textit{Multigrid} become capable of data exchange through API.
        \item This in turn makes easier \textit{reusability} of the extensions as building blocks.
    \end{itemize}
}


\end{subcolumns}

\end{columns}

% Block to hold the horizontal line
\useblockstyle{NoFrameBlockStyle}
\block{}{
    \textcolor{colorThree}{\rule{\linewidth}{1pt}} % Change 1pt to desired thickness
}
\useblockstyle{Square}

\begin{columns}

% 1st column
\column{0.3}
\block{Demonstration/Results}{
Preliminary results

...

...

...

...
}


% 2nd column
\column{0.35}
\block{Discussion}{
    In line with the theme proposed for RSECon24, My RSE work tries to follow some guiding principles of FAIR for Research Software:
    \begin{itemize}
        \item My implementation enables new ways of coupling solvers for different PDE systems which weren't possible before, thus improving \textit{interoperability}. Various Peano4 extensions, such as \textit{ExaHyPE} and \textit{Multigrid} become capable of data exchange through API.
        \item This in turn makes easier \textit{reusability} of the extensions as building blocks.
    \end{itemize}
}


% 3rd column
\column{0.35}
\block{?}{
  \begin{minipage}{.7\textwidth}
    \bibliographystyle{plain}
    \bibliography{biblio}
    %    Research supported by ...
  \end{minipage}
}

\end{columns}

% Footer
\footer[height=6cm]{
  % Needs a bit more space for the logos
%  \hspace{0.05cm}
  % Since we have two sponsors, UCalgary brand guidelines say we
  % should use wordmarks not lockups
%  \begin{minipage}{.5\textwidth}
%    \includegraphics[height=4cm]{ti-wordmark.pdf}
%    \hspace{.1\textwidth}
%    \includegraphics[height=4cm]{philosophy-wordmark.pdf}
%  \end{minipage}
  \hfill
  \begin{minipage}{.2\textwidth}
    \raggedleft \textbf{Contact:}\\
    Dmitry Nikolaenko\\
    dmitry.nikolaenko\@durham.ac.uk
  \end{minipage}
  %\hspace{0.05cm}
}

\end{document}


