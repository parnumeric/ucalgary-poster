\block{Introduction}{
    \textbf{Peano} is a \textit{framework} for PDE solvers on hierarchical meshes with adaptive mesh refinement \cite{ref1} and provides the mesh management, data storage, distribution and mesh traversal on the Peano \textit{space-filling curve}. Peano is massively parallelised and capable of exploiting the full resources of supercomputers. Its current version 4 is shipped with several \textit{specialised extensions} following Peano architecture as well as applications and benchmarks to demonstrate its work and assess its performance.
    %These extensions rely on the unified API provided by Peano4.
    \vspace{0.5in}

    \textbf{ExaHyPE2} is one of \underline{extensions} for solving systems of first-order hyperbolic partial differential equations (PDEs) \cite{REINARZ2020107251} (e.g., in problems of seismology and astrophysics). It provides a generic engine and collection of solvers (such as finite volume method and higher order ADER discontinuous Galerkin schemes) to solve systems of PDEs.
    \vspace{0.5in}

    %\textbf{Swift2} is another \underline{extension} for solving astrophysical problem by means of Smoothed Particle Hydrodynamics (SPH) method. It integrates the SPH kernels from SWIFT software employing task parallelism with Peano's particle-in-dual-tree concept and its particles toolbox.
    %\vspace{0.5in}

    \textbf{Multigrid} is another \underline{extension} being developed in collaboration with mathematicians from Bath University. It implements elliptic solvers based on multigrid methods using a hierarchy of discretisations.
    \vspace{0.5in}

    Every Peano application, including ones based on the above extensions, is a C++ code which follows the unified Peano architecture consisting of several layers:
    \begin{itemize}
        \item Technical Architecture \texttt{<<tarch>>}
        \item Peano core \texttt{<<peano4>>}
        \item Various toolboxes, on top of which extensions are built, e.g.:
        \begin{itemize}
            \item \texttt{<<toolbox::blockstructured>>} for blockstructured meshes
            \item \texttt{<<toolbox::particles>>} for particle management
            \item \texttt{<<toolbox::loadbalancing>>} for dynamic load balancing
        \end{itemize}
    \end{itemize}
}