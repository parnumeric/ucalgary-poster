\block{Introduction}{
    \textbf{Peano} is a massively parallel PDE solver with hierarchical meshes and adaptive mesh refinement \cite{ref1}. Peano provides a \textit{framework} for dynamically \textit{adaptive
    Cartesian meshes} which are traversed based on the Peano \textit{space-filling curve}. Peano is parallelised and capable of exploiting the full resources of supercomputers. Peano4 (in its current version 4) is shipped with several \textit{specialised extensions} following Peano4 architecture as well as application and benchmarks demonstration  work.
    %These extensions rely on the unified API provided by Peano4.
    \vspace{0.5in}

    \textbf{ExaHyPE2} is one of \underline{Peano4 extensions}  for solving systems of first-order hyperbolic partial differential equations (PDEs) \cite{REINARZ2020107251} (e.g., in problems of seismology and astrophysics). It provides a generic engine and collection of solvers (such as finite volume method and higher order ADER discontinuous Galerkin schemes) to solve systems of PDEs.
    \vspace{0.5in}

    %\textbf{Swift2} is another \underline{Peano4 extension} for solving astrophysical and cosmological problem by means of Smoothed Particle Hydrodynamics (SPH) method. It integrates the SPH kernels from SWIFT software employing task parallelism with Peano's particle-in-dual-tree concept and its particles toolbox.
    %\vspace{0.5in}

    \textbf{Multigrid} is another \underline{Peano4 extension} being developed in collaboration with mathematicians from Bath University. It implements elliptic solvers based on multigrid methods using a hierarchy of discretisations.
    \vspace{0.5in}

    Every Peano4 application, including ones based on the above extensions, is a C++ code which follows the unified Peano4 architecture consisting of several layers:
    \begin{itemize}
        \item Technical Architecture \texttt{<<tarch>>}
        \item Peano4 core \texttt{<<peano4>>}
        \item Various toolboxes, on top of which extensions are built, such as:
    \begin{itemize}
        \item \texttt{<<toolbox::blockstructured>>} for blockstructured meshes
        \item \texttt{<<toolbox::particles>>} for particle management
        \item \texttt{<<toolbox::loadbalancing>>} for dynamic load balancing
    \end{itemize}
    \end{itemize}
}